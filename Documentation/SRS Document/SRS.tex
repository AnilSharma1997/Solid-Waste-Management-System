%Format: Latex
\documentclass{article}
\setcounter{secnumdepth}{5}
\setlength{\textwidth}{17cm}
\setlength{\textheight}{9in}
\setlength{\topmargin}{-1.5cm}
\setlength{\oddsidemargin}{0in}
\setlength{\evensidemargin}{0in}
\usepackage{textcomp}
\usepackage{booktabs}
\usepackage{fancyhdr}
\usepackage{times}
\usepackage{tikz}
\usepackage{amsmath}
\usepackage{tabulary}
\usepackage{pgfgantt}
\usepackage[utf8]{inputenc}
\usepackage{tikz}
\usetikzlibrary{shapes.geometric, arrows}

\tikzstyle{arrow} = [thick,->,>=stealth]

\usepackage{verbatim}
\usetikzlibrary{arrows,shapes}
\usepackage{adjustbox}
\usepackage{forest}
\usepackage{tikz-qtree}
\usepackage{soul}
\pagestyle{fancy}
\rhead{CSE\hspace{\labelsep}\textbullet\hspace{\labelsep} Autumn 2017}
\lhead{\textbf{CS302} Software Engineering}
\cfoot{\thepage}
\usepackage{indentfirst}
 \usepackage{graphicx}
\graphicspath{ {Images/} }
\setcounter{tocdepth}{5} 
\usepackage{hyperref}
\hypersetup{
    colorlinks=true,
    linkcolor=blue,
    filecolor=blue,      
    urlcolor=blue,
}
\title{\textbf{CS302}\\\HUGE Software Engineering\\
\LARGE CSE\hspace{\labelsep}\textbullet\hspace{\labelsep} Autumn 2017
}

\author{Hexagineers}



\begin{document}
\maketitle
\line(1,0){450}

\begin{center}
\textbf{\Huge Lorem Ipsum Dolor\\\Large Requirement Analysis}

\end{center}
\newpage
\tableofcontents
\newpage

\section{Introduction}
\subsection{Purpose}
The  purpose  of  this  Software  Requirements  Specifica
tion  (SRS)  document  is  to  provide  a 
detailed description of the functionalities of the 
\textbf{Lorem Ipsum Dolor} system. This document will cover each of  the  system’s intended  features,  as well  as  offer a  preliminary  glimpse of  the  software 
application’s User Interface (UI). The document will also cover hardware, software, and various other technical dependencies. 

\subsection{Intended Audience} 
\par his  document  is  intended  for  all the individuals  participating  in  and/or supervising  the  \textbf{Lorem Ipsum Dolor project}. 
This document provides a brief overview of each aspect of the project as a whole. 

\subsection{Project Scope}
\par Lorem Ipsum DOlor is composed of two main components: a web-app for users tu buy products online and track their profiles, and an app for the \textit{kabadiwalas} to enable them with technology that will ease their work of collection of kabad from various households from various localities.
\newpage

\section{Overall Description}
\subsection{Product Perspective}
\par \textbf{Lorem Ipsum Dolor} is a new product intended for use on the Android platform and web browsers. As discussed earlier, the project will have two platforms. An android app for the kabadiwalas to enable them with the means to quickly reach households where solid waste is ready for disposal. The other will be a web-app which will enable the customers to buy products and keep track of their activity and profiles.
\subsection{Product Features}
\par The core features of the projects are described below. 
\begin{itemize}
    \item \textbf{Credit-Based System}
    \par All the customers who give away the solid waste would be given credits which they could redeem online.
    These credits would also depend upon what kind of waste and how much amount of waste is given away.
    
    \item \textbf{Sale of Recycled Items}
    \par The e-commerce website would have a range of recycled goods (like, decorative pieces, sofas, leather items, fiber bags etc). Customers can choose what they like from this site.
    
    \item \textbf{Reusing Videos}
    \par This section will comprise of know-how videos on how to reuse various solid waste lying around the household to make something creative or useful. These videos can include guides on how to make recycled paper greeting cards, clocks, decorative items etc. 
    
    \item \textbf{Waste Pick-up}
    \par The customers can schedule the solid waste pick up. This will enable the customer to get rid of the wste quickly and not wait for the Kabadiwala to come. 
    
    \item \textbf{Special app for Kabadiwalas}
    \par We will provide the kabadiwalas with a special app that will be easy to use. This app will notify when there is a customer who wants to dispose off his waste. Such technology will enable to work efficiently and go to only those places where there is a necessity.
    
    
\end{itemize}
\subsection{User Classes and Characteristics}
\par The \textbf{Lorem Ipsum Dolor} is supposed to address the issue of solid waste management pan-India. We believe that the project will belong to the following classes\footnote{\textbf{Classification of Cities} by Government of India, Ministry of Finance, Department of Expenditure\\\href{http://www.finmin.nic.in/sites/default/files/21-07-2015_0.pdf}{Memorandum \textbf{No.2/5/2014-E.II(B)}}}:

\begin{itemize}
    \item \textbf{Tier 1 (X)}
    \par Dry Waste Management is a burning issue in big, metro cities. The sole cause of this is because the population in a metro city is comparatively high. Waste generated in such cites are higher from both commercial and non-commercial sources.
    
    \item \textbf{Tier 2 (Y)}
    \par The waste generated is comparatively lesser than Tier 1 (X) cities. This becomes the second class of cities. The success of project will depend upon the impact created in the Tier 1 cities.
    
    \item \textbf{Tier 3 (Z)}
    \par These cities will be at the least focus since there will not be much business opportunity in the initial years of the project. However, on a long term, the success in these cities will, again, depend upon the success of the idea in the higher tier cities.
\end{itemize}
\par These classification are not meant to separate public into different groups. However, this will be the flow in which the project will be made accessible to the public (from Tier 1 to Tier 3). 
\par It  is  crucial  that  each  of  these  tiers  be  fully  supported  in  the  final  product  so  as  to maximize the overall value of the product. It is also important that the application be as user-friendly as possible, otherwise it will lack the impact which the product desires to create. Most importantly, the application must be reliable. Regardless of the situation, the application must accurately perform the functionality that were mentioned in the above section.  
\subsection{Operating Environment}
\par The main component of the project is the software application, which will be limited to the android platform-based app and web browsers. The application is not 
resource- or graphics-intensive, so there are no practical hardware constraints. The app(android and web app) will rely on  several  functionality. 
 
\subsection{Design and Implementation Constraints}
\par The primary design constraint is the mobile platform and the browser platform. Since the application is essentially designated for mobile  handsets, limited screen size and resolution will be a major design consideration. Creating a user interface  which is both effective and easily navigable will pose a difficult challenge. Since the project involves collection of actual waste, e-commerce platform to sell products on a large scale, each feature must be designed and implemented with efficiency in mind. 
\subsection{Assumption and dependencies}
\par
\end{document}
