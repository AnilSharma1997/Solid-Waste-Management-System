%Format: Latex
\documentclass{article}
\setlength{\textwidth}{17cm}
\setlength{\textheight}{9in}
\setlength{\topmargin}{-1.5cm}
\setlength{\oddsidemargin}{0in}
\setlength{\evensidemargin}{0in}
\usepackage{textcomp}
\usepackage{booktabs}
\usepackage{fancyhdr}
\usepackage{times}
\usepackage{tikz}
\usepackage{amsmath}
\usepackage{tabulary}
\usepackage{pgfgantt}
\usepackage[utf8]{inputenc}
\usepackage{tikz}
\usetikzlibrary{shapes.geometric, arrows}

\tikzstyle{arrow} = [thick,->,>=stealth]

\usepackage{verbatim}
\usetikzlibrary{arrows,shapes}
\usepackage{adjustbox}
\usepackage{forest}
\usepackage{tikz-qtree}
\usepackage{soul}
\pagestyle{fancy}
\rhead{CSE\hspace{\labelsep}\textbullet\hspace{\labelsep} Autumn 2017}
\lhead{\textbf{CS302} Software Engineering}
\cfoot{\thepage}
\usepackage{indentfirst}
 
\setcounter{tocdepth}{5} 
\usepackage{hyperref}
\hypersetup{
    colorlinks=true,
    linkcolor=blue,
    filecolor=blue,      
    urlcolor=blue,
}
\title{\textbf{CS302}\\\HUGE Software Engineering\\
\LARGE CSE\hspace{\labelsep}\textbullet\hspace{\labelsep} Autumn 2017
}

\author{Hexagineers}



\begin{document}

\maketitle
\line(1,0){450}

\begin{center}
\textbf{\Huge Feasibilty Report}
\end{center}
\line(1,0){450}
\newpage

\tableofcontents
\newpage

\section{Preface}
\par The objective of this document is to provide the reader with all the possible ideas that we came up with and the idea we are working on.

\par This document states the problem statement and analyses each ideas on the basis of three aspects. \textbf{(1) Market Research}, \textbf{(2) Time Feasibility} and \textbf{(3) Scalibility}. 
\par The members working on this project are:
\begin{itemize}
    \item Akash Agrawal
    \item Anil Sharma
    \item Megana Ganapathiraju
    \item Madhukar Jaiswal
    \item Vikesh Meena
    \item Tanmay Khandait
\end{itemize}
\newpage
\section{List of Possible Ideas}
\par While discussing about the possible project ideas, all the project members seemed to be keen on doing something that can solve a pressing social issue in India. Keeping this in mind, we came across certain ideas in various sectors.
\subsection{Health Care System}
\subsubsection{Problem Statement}
\par India is moving towards digitization at a pace never seen before. However, we lack the necessary digitization of health care system. 
We are talking about creating a network where the doctors, patients and all the related components of the health care system (like diagnostic labs, pharmacies etc.,) are integrated together on a single network system.

\subsubsection{Feasibility of the Idea}
\begin{itemize}
\item \textbf{Field Research}
\par The stakeholders involved are as follows
\begin{itemize}
    \item Doctors
    \item People
    \item Pharmaceutical Stores
    \item Diagnostic Labs
    \item Hospital Authorities
    \item Government
\end{itemize}
This project demands intense amount of interaction with the stakeholders to convince the stakeholders to be a part of this idea. This project also demands approval from a lot of government authorities, hospital authorities and their openness towards this idea

\item \textbf{Time Feasibility}
\par Considering the scope of this project, a lot of time has to spent with the stakeholders to understand their demands from this network. Market research will alone take a couple of months, only after which we could move towards implementation of this idea. Overall the project involves a lot of groundwork that is not feasible within a limited span of time.
\item \textbf{Scalibility}
\par This project is in line with the current development in India. With digitization at it's peak, this idea will gain a lot of support in plans like \textbf{Smart City Project}. It wouldn't be wrong to say that the scalabilty of this idea is immensely high.
\end{itemize}


\subsection{Android-based Game}
\subsubsection{Problem Statement}
\par In today's age, mobile phones have pierced through all the age groups. Kids, today, are heavily influenced by mobile phone games and apps. Keeping this fact in mind, is it possible to create a game wherein we could inculcate in the target age-group the spirit of saving environment. 

\subsubsection{Feasibility of the Idea}
\begin{itemize}
    \item \textbf{Market Research}
    \par The stakeholders involved in this project are :
    \begin{itemize}
        \item People
    \end{itemize}
    \par This project involves research related to the plot we want to frame the game around. The following things are to be considered while developing the game
        \begin{itemize}
            \item The target age -group
            \item The story-line of the game
            \item The extent of graphics in the game.
        \end{itemize}
    
    \item \textbf{Time Feasibility}
    \par Developing an android-based game involves expertise in graphic designing, character development, plot development,
and choosing (or developing) a game engine.  With these skills just being a starter for basic game development, the fact
that none of us have the necessary skill set makes it tougher to implement in such a short period of time.
\item \textbf{Scalibility}
\par The scope of the project is limited to the users who play the game. However, with subject to taste of the audience and plot of the game, the impact of the game could be judged.
\end{itemize}


\subsection{Railways: Automation of TTE Work}
\subsubsection{Problem Statement}
\par The government is spending a lot on Railway Infrastructure. The government has introduced the Bullet Train, Taigo etc., and is also focusing a lot of money and time on upgrading the existing railway system. While such major developments take place, is there a way where we could digitize the work of the TTE, i.e., automatically allot seats to travellers on board in situation like (Wait List, RAC etc.) This will result in better experience to the on-board travellers and also reduce illegal and unethical practices.

\subsubsection{Feasibility of the Idea}

\begin{itemize}
    \item \textbf{Market Research}
    \par The stakeholders involved in this project are:
    \begin{itemize}
        \item Railway System
        \item TTE
        \item General Public
        \item Government
    \end{itemize}
    \par This project will result in a total digitization of the on-board interaction with TTE. This idea involves intense amount of:
    \begin{itemize}
        \item \textbf{Interaction} with the stakeholders in order to understand their expectation from this project. Since this idea involves partial lose of jobs, we believe that this idea would not be appreciated by the authorities.
        \item \textbf{Research} how the work of the TTE is carried out. We will have to research not just the way TTE functions, but also understand the way the Railway system works (like IRCTC, online booking, offline booking etc.). Definitely, all these would involve seeking permission with lots of authorities and would involve a huge amount of time spent in groundwork, much more than the span of our project work. Moreover we are not sure that we would get these internal information about the working of railway authorities. If we do not get the detailed information then it will be extremely difficult to implement this project plan. 
    \end{itemize}
    \item \textbf{Time Feasibility}
    \par Considering the scope of this project, a lot of time has to be spent with the stakeholders to understand their demands from this project. As mentioned, this idea will involve an immense amount of interaction with stakeholders and research of the existing system. 
    \item \textbf{Scalibility}
    \par This project is in line with the current development in India. With digitization at it's peak, this idea will gain a lot of support in plans like \textbf{Smart City Project}. With a majority of expenditure being invested on bringing new systems and also upgrading the existing the systems, the scope of this project is very high.
    
\end{itemize}
\subsection{Solid Dry Waste Reusing and Recycling}
\subsubsection{Problem Statement}
India generates around tonnes of solid dry waste, better known as \textit{kabad}. The idea of this project is to use this dry waste and provide people with recycled goods made from their own waste that they gave away. This project will also focus on educating people on how to reuse and make objects of decoration or purpose at home from the waste. Broadly speaking, this project will promote the idea of reusing and recycling goods. All together this is in sync with the current urge of making a clean India. 

\subsubsection{Feasibility of the Idea}
\begin{itemize}
    \item \textbf{Market Research}
    \par The stakeholders of this project are:
    \begin{itemize}
        \item People
        \item Recycling Industries
        \item Kabadiwalas
        \item Government
        \item Transport agency
    \end{itemize}
    The main idea of this project is to collect the \textit{kabad} from the people in exchange for credits which they could redeem from our e-commerce store to buy recycled goods. These goods will include some basic items like recycled paper, decoration items etc. The credits issued would be based on the value of goods that the customers sell to the scrap dealers. Apart from this, there will also be a section where people could learn, through online videos, about making goods by reusing various things at home. This project will hence involve interaction on both the ends of the supply chain, i.e., the customers as well as the recycling industries. We even need to interact with the concerned government authorities to obtain their permissions to collect the waste as well as sell recycled products.
    \item \textbf{Time Feasibility}
    \par This project will take around a couple of weeks to interact with people as well as the industries to understand the process of collection as well as recycling. The major hurdle will be the logistics of the goods. Technically, this idea will involve implementation of a website for the customers and an app for the kabadiwalas to record the collection of waste. However, estimating the time it will take for this idea to circulate amongst the people cannot be perfectly estimated given the time frame.
    \item \textbf{Scalibility}
    With \textbf{Swachh Bharat Abhiyan} and several other cleanliness drives being organized around the country, this idea sure seems to be sustainable. People today have also switched to online shopping.Considering these facts and the support of the government towards cleanliness and digitization, the scalability of the idea is immensely high.
    
\end{itemize}

\section{Our Project : Solid Dry Waste Reusing and Recycling}
\par We chose this idea because amongst all the ideas because:
\begin{itemize}
    \item The concept seemed pretty new and fancy. From a business point-of-view, this idea could be further carried on to evolve into a start-up. 
    \item In terms of technical skills, our team members possess all the relevant skills to do this project.
    \item The time required to implement the ideas satisfy our deadline constraints.
\end{itemize} 
\end{document}
