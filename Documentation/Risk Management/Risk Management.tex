%Format: Latex
\documentclass{article}
\setcounter{secnumdepth}{5}
\setlength{\textwidth}{17cm}
\setlength{\textheight}{9in}
\setlength{\topmargin}{-1.5cm}
\setlength{\oddsidemargin}{0in}
\setlength{\evensidemargin}{0in}
\usepackage{textcomp}
\usepackage{booktabs}
\usepackage{fancyhdr}
\usepackage{times}
\usepackage{tikz}
\usepackage{amsmath}
\usepackage{tabulary}
\usepackage{pgfgantt}
\usepackage[utf8]{inputenc}
\usepackage{tikz}
\usetikzlibrary{shapes.geometric, arrows}
\usepackage{multirow}
\usepackage[table]{xcolor}
\tikzstyle{arrow} = [thick,->,>=stealth]
\usepackage{colortbl}
\usepackage{verbatim}
\usetikzlibrary{arrows,shapes}
\usepackage{adjustbox}
\usepackage{forest}
\usepackage{tikz-qtree}
\usepackage{soul}
\pagestyle{fancy}
\rhead{CSE\hspace{\labelsep}\textbullet\hspace{\labelsep} Autumn 2017}
\lhead{\textbf{CS302} Software Engineering}
\cfoot{\thepage}
\usepackage{indentfirst}
 \usepackage{graphicx}
\graphicspath{ {Images/} }
\setcounter{tocdepth}{5} 
\usepackage{hyperref}
\hypersetup{
    colorlinks=true,
    linkcolor=blue,
    filecolor=blue,      
    urlcolor=blue,
}
\title{\textbf{CS302}\\\HUGE Software Engineering\\
\LARGE CSE\hspace{\labelsep}\textbullet\hspace{\labelsep} Autumn 2017
}

\author{Hexagineers}



\begin{document}
\maketitle
\line(1,0){450}

\begin{center}
\Huge\textbf{Solid Waste Management System}\\
\Large \textbf{Risk Management Plan}

\end{center}
\newpage
\tableofcontents
\newpage
\section{Introduction}
\subsection{Purpose}
A risk is an event or condition that, if it occurs, could have a positive or negative effect on a project’s objectives. Risk Management is the process of identifying, assessing, responding to, monitoring, and reporting risks. This Risk Management Plan defines how risks associated with this project will be identified, analyzed, and managed. It outlines how risk management activities will be performed, recorded, and monitored throughout the lifecycle of the project and provides templates and practices for recording and prioritizing risks.
The Risk Management Plan monitored and updated throughout the project.  
The intended audience of this document is the project team, project sponsor and management.

\section{Risk Management Procedure}
\subsection{Process}
The project manager working with the project team and project sponsors will ensure that risks are actively identified, analyzed, and managed throughout the life of the project.  Risks will be identified as early as possible in the project so as to minimize their impact.  The steps for accomplishing this are outlined in the following sections.

\subsection{Risk Identification}
Risk identification will involve the project team, appropriate stakeholders, and will include an evaluation of environmental factors, organizational culture and the project management plan including the project scope.  Careful attention will be given to the project deliverables, assumptions, constraints and deadlines.

\subsection{Risk Analysis}
All risks identified will be assessed to identify the range of possible project outcomes.  Qualification will be used to determine which risks are the top risks to pursue and respond to and which risks can be ignored.

\subsubsection{Qualitative Risk Analysis}
The probability and impact of occurrence for each identified risk will be assessed by the team, with input from the team itself, along with the stakeholders by using the following approach:

\begin{itemize}
    \item \textbf{Probability}
    \begin{itemize}
         \item High – Greater than 70\% probability of occurrence.
    \item Medium – Between 30\% to 70\% probability of occurrence.
    \item Low – Below 30\% probability of occurrence.
    \end{itemize}
    \item \textbf{Impact}
    \begin{itemize}
        \item High – Risk that has the potential to greatly impact project cost, project schedule or performance
        \item Medium – Risk that has the potential to slightly impact project cost, project schedule or performance
        \item Low – Risk that has relatively little impact on cost, schedule or performance
    \end{itemize}
    \begin{table}[h]
\centering
\begin{tabular}{|c|c|lll|}
\cline{1-5}
\multirow{5}{*}{\textbf{Impact}} & \textbf{H} & \cellcolor{yellow} &\cellcolor{red}  & \cellcolor{red} \\ \cline{2-2}
 & \textbf{M} & \cellcolor{green} & \cellcolor{yellow} &\cellcolor{red}  \\ \cline{2-2}
 & \textbf{L} &  \cellcolor{green}& \cellcolor{green} &\cellcolor{yellow}  \\ \cline{2-5} 
 & \textbf{} & \multicolumn{1}{l|}{\textbf{L}} & \multicolumn{1}{l|}{\textbf{M}} & \multicolumn{1}{l|}{\textbf{H}} \\ \cline{2-5} 
 & \multicolumn{4}{c|}{\textbf{Probability}} \\ \hline
\end{tabular}
\end{table}
Risks that fall within the RED and YELLOW zones will have risk response planning which may include both a risk mitigation and a risk contingency plan.
\end{itemize}
\subsubsection{Quantitative Risk Analysis}
Analysis of risk events that have been figured out using the qualitative risk analysis process and their affect on project activities will be guessed. A numerical rating will then be applied to each risk based on this analysis, and then documented in this section of the risk management plan.

\subsection{Risk Response Planning}
Each major risk (those falling in the Red & Yellow zones) will be assigned to a project team member for monitoring purposes to ensure that the risk will not “fall through the cracks”.  
For each major risk, one of the following approaches will be selected to address it:
\begin{itemize}
    \item \textbf{Avoid} – eliminate the threat by eliminating the cause
    \item \textbf{Mitigate} – Identify ways to reduce the probability or the impact of the risk
    \item \textbf{Accept} – Nothing will be done 
    \item \textbf{Transfer} – Make another party responsible for the risk (buy insurance, outsourcing, etc.)
\end{itemize}

For each risk that will be mitigated, the project team will identify ways to prevent the risk from occurring or reduce its impact or probability of occurring.  This may include prototyping, adding tasks to the project schedule, adding resources, etc.
For each major risk that is to be mitigated or that is accepted, a course of action will be outlined for the event that the risk gets managed in order to minimize its impact.
\subsection{Risk Monitoring, Controlling and Reporting}
The level of risk on a project will be tracked, monitored and reported throughout the project life-cycle.  
\section{Classification and Analysis of Risks}
\subsection{Software Requirement Risks}
\begin{itemize}
    \item \textbf{Poor definition of requirements}
    \item \textbf{Lack of analysis of requirements}
    \item \textbf{Ambiguity of requirements}
    \item \textbf{Inadequacy of requirements}
    \item \textbf{Changes in requirements}
\end{itemize}
\subsection{Software Cost Risks}
\begin{itemize}
    \item \textbf{Lack of good estimates}
    \item \textbf{Unrealistic schedule}
    \item \textbf{Human Errors}
    \item \textbf{Lack of monitoring}
    \item \textbf{Lack of testing}
    \item \textbf{Complexity of Software}
    \item \textbf{Failures of tools(dependencies)}
    \item \textbf{Personnel Change}
    \item \textbf{Disagreement between members}
    \item \textbf{Disagreement between client and team}
    \item \textbf{Management Change}
    \item \textbf{Technology Change}
    \item \textbf{Shortage of personnel}
    \item \textbf{Working space/Environment Change}
\end{itemize}
\subsection{Software Scheduling Risks}
\begin{itemize}
    \item \textbf{Inadequate Budget}
    \item \textbf{Changes or Extension of requirements}
    \item \textbf{Lack of skills}
    \item \textbf{Long term training of team}
    \item \textbf{Lack of experienced manager}
    \item \textbf{Inadequate knowledge about tools and techniques}
    \item \textbf{Difficulty of implementation}
    \item \textbf{Lack of skills}
\end{itemize}

\subsection{Software Quality Risks}
\begin{itemize}
    \item \textbf{Lack of project standard }
    \item \textbf{Lack of design documentation }
    \item \textbf{Inadequate budget}
    \item \textbf{Failure of Tools} (various third-part dependencies)
    \item \textbf{Changes in technology}
    \item \textbf{Failure of technology}
    \item \textbf{Loss of technical equipment}
    \item \textbf{Weakness of management}
    \item \textbf{Lack of collaboration by developer}
    \item \textbf{Human errors}
\end{itemize}

\subsection{Software Business Risks}
\begin{itemize}
    \item \textbf{No one wants the product}
    \item \textbf{Designed product is not in line with the desired product}
    \item \textbf{Product is tough to use}
    \item \textbf{Product is tough to sell}
    \item \textbf{Important stakeholders back off}
    \item \textbf{Failure to provide good customer experience}
\end{itemize}
\end{document}
